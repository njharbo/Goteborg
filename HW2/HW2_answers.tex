\documentclass{scrartcl}
\usepackage{multirow}
\usepackage{amsmath}
\usepackage{amssymb}
\usepackage{mathrsfs}
\usepackage{graphicx}
\usepackage{tikz}
\usetikzlibrary{calc}
\usepackage{float}
%\usepackage{undertilde}
%\usepackage{mathrsfs}
\usepackage{epstopdf}
\usepackage{listings}
\usepackage{tabu}
\usepackage{caption}
\usepackage{subcaption}
\usepackage{booktabs,caption,fixltx2e}
\usepackage[flushleft]{threeparttable}
%%\usepackage{courier}
\lstset{basicstyle=\footnotesize\ttfamily,breaklines=true}
\usepackage{booktabs}
\usepackage{fancyhdr}
\usepackage[colorlinks=true, hidelinks]{hyperref}
\usepackage[tmargin=1.1in,bmargin=1.1in,lmargin=1.45in,rmargin=1.45in]{geometry}

%% User defined commands
\newcommand{\ra}[1]{\renewcommand{\arraystretch}{#1}}
	% Integrals
\newcommand*\diff{\mathop{}\!\mathrm{d}}
	% Symbols
\newcommand{\RR}{\mathbb{R}}
\newcommand{\LL}{\mathcal{L}}
	% Operators
\newcommand{\Max}[2]{\max_{#1} \left\{ #2 \right\} } % Maximum
\newcommand{\Min}[2]{\min_{#1} \left\{ #2 \right\} } % Minimum
\newcommand{\Argmax}[2]{\argmax_{#1} \left\{ #2 \right\} }
\newcommand{\Argmin}[2]{\argmin_{#1} \left\{ #2 \right\} }
\newcommand{\Ln}[1]{\ln{ \left( #1 \right)} }
\newcommand{\Sumt}{\sum_{t=0}^\infty }
	% Statistics
\newcommand{\E}[1]{\text{E} \left[ #1 \right]}
\newcommand{\Var}[1]{\text{Var} \left( #1 \right)}
\newcommand{\Cov}[1]{\text{Cov} \left( #1 \right)}
\newcommand{\Y}{\utilde{Y}}
\newcommand{\X}{\utilde{X}}
\newcommand{\Z}{\utilde{Z}}
\newcommand{\e}{\utilde{e}}
\newcommand{\ML}[1]{\hat{1}_{\text{ML}}}
	% Parantheses
		% Normal
\newcommand{\Le}{\left(}
\newcommand{\Ri}{\right)}
\newcommand{\Lep}{\left(}
\newcommand{\Rip}{\right)}
\newcommand{\lep}{\left(}
\newcommand{\rip}{\right)}
		% Curly
\newcommand{\Lec}{\left\{}
\newcommand{\Ric}{\right\}}
\newcommand{\lec}{\left\{}
\newcommand{\ric}{\right\}}
		% Hard
\newcommand{\Leh}{\left[}
\newcommand{\Rih}{\right]}
\newcommand{\leh}{\left[}
\newcommand{\rih}{\right]}
% commands
\newcommand{\bal}{\begin{align}}
\newcommand{\eal}{\end{align}}
\newcommand{\bit}{\begin{itemize}}
\newcommand{\eit}{\end{itemize}}

\definecolor{steelblue}{RGB}{64,134,170}

\begin{document}

\pagestyle{fancy}
\fancyhead{}
\fancyhead[LE,RO]{\thepage}
\lhead{Macro 1 2015, Homework 2}
\fancyfoot{}

\title{Macroeconomics I}
\subtitle{Homework 2 - Suggested solutions}
\author{Niels-Jakob Harbo Hansen}
\maketitle

\section*{Introduction}
If you find any errors or unclear sections in those answers, please help me and your classmates by emailing me at Niels-Jakob Harbo Hansen at \url{nielsjakobharbo.hansen@iies.su.se} or Jonna Olsson \url{jonna.olsson@ne.su.se},. Please also send any of us an email if you have other questions! 

Your homeworks will be handed back in the next seminar. If you want to get feed-back earlier, just send me an email. 

\section*{Problem 1}

\subsection*{Part (a)}

\begin{itemize}
\item A \textbf{sequential competitive equilibrium} is a set of sequences for allocations $\left\{a_{1t}^*, c_{1t}^*\right\}_{t=0}^{\infty}$ and $\left\{a_{2t}^*, c_{2t}^*\right\}_{t=0}^{\infty}$ and prices $\left\{q_{t}^*\right\}_{t=0}^{\infty}$ such that 
	
	\begin{enumerate}
		\item $\left\{a_{it}^*, c_{it}^*\right\}_{t=0}^{\infty}$ solves the problem of the household:
		\begin{align}
		&\max_{\{{c_{i,t}, a_{i,t+1}}\}_{t=0}^{\infty}} {\sum_{t=0}^{\infty} \beta^t u(c_t)} \text{ for i=1,2}\\
		&\text{s.t. } \nonumber \\ 
		&c_{it}+q_t^* a_{i,t+1}=e_{it}+a_{it} \forall t \nonumber \\
		& \lim_{t \rightarrow \infty} \frac{a_{t+1}}{(1+r)^t} \geq 0 \nonumber 
		\end{align}
		\item Markets clear
		\begin{align}
		\sum_{i} c_{it}^*&=\sum_{i} e_{it} \text{ } &\forall t \\
		\sum_{i} a_{it}^*&=0 \text{ } &\forall t 
		\end{align}
	\end{enumerate}
	
\end{itemize}

\subsection*{Part (b)}

\begin{itemize}
	\item A \textbf{date-zero competitive equilibrium} is a set of sequences for allocations $\left\{c_{1t}^*\right\}_{t=0}^{\infty}$ and $\left\{c_{2t}^*\right\}_{t=0}^{\infty}$ and prices $\left\{p_{t}^*\right\}_{t=0}^{\infty}$ such that 
	
	\begin{enumerate}
		\item $\left\{c_{it}^*\right\}_{t=0}^{\infty}$ solves the problem of the household:
		\begin{align}
		&\max_{\{{c_{it}}\}_{t=0}^{\infty}} {\sum_{t=0}^{\infty} \beta^t u(c_t)} \text{ for i=1,2}\\
		&\text{s.t. } \sum_t p_{t} c_{it}=\sum_t p_{t} e_{it} \nonumber 
		\end{align}
		\item Markets clear
		\begin{align}
		\sum_{i} c_{it}^*&=\sum_{i} e_{it}^* \text{ } \forall t
		\end{align}
	\end{enumerate}

\end{itemize}

\subsection*{Part (c)}

\subsubsection*{Sequential equilibrium}

\begin{itemize}
	\item The Lagrangean for the household problem reads
	\begin{align}
	L=\sum_{t} \beta^t u_{i} (c_{it})+\sum_t \lambda_{it} \left( c_{it}+q_t a_{i,t+1}-e_{it}-a_{it}\right)
	\end{align}
	\item The first order conditions become
	\begin{align}
	c_{it}: \beta^t u^\prime_i(c_it)+\lambda_{it}=0 \\
	a_{i,t+1}: \lambda_{it} q_t - \lambda_{it+1}=0
	\end{align}
	\item Combine to get
	\begin{align}
	\frac{q_t}{\beta}=\frac{u^\prime(c_{i,t+1})}{u^\prime(c_{i,t})} \text{ } \forall i,t \label{eq:q}
	\end{align}
	\item This implies 
	\begin{align}
	\frac{u^\prime(c_{1,t+1})}{u^\prime(c_{1,t})}&=	\frac{u^\prime(c_{2,t+1})}{u^\prime(c_{2,t})} \\
	\Rightarrow c_{i,t+1}&=c_{i,t}=c_i
	\end{align}
	\item Inserting into \eqref{eq:q} yields 
	\begin{align}
	q_t^*=\beta
	\end{align}
	\item Now note that the consolidated budget constraint, $\sum_t \left( \beta \right)^t c_{it}=\sum_t \left( \beta \right)^t e_{it}$ for agent 1 reads
	\begin{align}
	\sum_t \left( \beta \right)^t c_{1}&=e_h+\left( \beta \right)e_l+\left( \beta \right)^2e_h+\dots \\
	c_{1}\frac{1}{1-\beta}&=e_h\frac{1}{1-\beta^2}+e_l\beta\frac{1}{1-\beta^2}
	\end{align}
\item And for agent 2
	\begin{align}
	\sum_t \left( \beta \right)^t c_{2}&=e_l+\left( \beta \right)e_h+\left( \beta \right)^2e_l+\dots \\
	c_{2}\frac{1}{1-\beta}&=e_l\frac{1}{1-\beta^2}+e_h\beta\frac{1}{1-\beta^2}
	\end{align}
	\item Hence,
		\begin{align}
		c_{1}=e_h\frac{1-\beta}{1-\beta^2}+e_l\frac{\beta(1-\beta)}{1-\beta^2} \\
		c_{2}=e_l\frac{1-\beta}{1-\beta^2}+e_h\frac{\beta(1-\beta)}{1-\beta^2} 
		\end{align}
	
	\item Verify that this indeed is an equilibrium by checking whether $c_1$ and $c_2$ sum to $e_h+e_l$.
	
	\begin{align}
	c_1+c_2&=e_h\frac{1-\beta}{1-\beta^2}+e_l\frac{\beta(1-\beta)}{1-\beta^2}+e_l\frac{1-\beta}{1-\beta^2}+e_h\frac{\beta(1-\beta)}{1-\beta^2} \nonumber \\
	&= e_h \frac{1-\beta+\beta(1-\beta)}{1-\beta^2}+e_l \frac{1-\beta+\beta(1-\beta)}{1-\beta^2} \nonumber \\
	&=e_h+e_l
	\end{align}
		
\end{itemize}

\subsubsection*{Date-zero equilibrium}

\begin{itemize}
	\item The Lagrangean reads
	
		\begin{align}
	L=\sum_{t} \beta^t u_{i} (c_{it})+ \lambda_{i} \sum_t \left( p_t c_{it}-p_t e_{it}\right)
	\end{align}
	
	\item Yields the first order condtions
	\begin{align}
	c_{1t}: \beta^t u^{\prime}(c_{1t})-\lambda_1 p_t=0  \text{ } \forall t\\
	c_{2t}: \beta^t u^{\prime}(c_{2t})-\lambda_2 p_t=0 \text{ } \forall t
	\end{align}
	
	\item From this we get
	
	\begin{align}
	\frac{u^{\prime}(c_{1,t+1})}{u^{\prime}(c_{1,t})}=\frac{u^{\prime}(c_{2,t+1})}{u^{\prime}(c_{2,t})}=\frac{p_{t+1}}{\beta p_{t}}
	\end{align}
	
	\item By similar argument from above we then get
	\begin{align}
	c_{i,t}=c_{i,t+1}
	\end{align}
	\item And then
	\begin{align}
	\frac{p_{t+1}}{p_{t}}=\beta
	\end{align}
	
	\item Now use this in agent 1's budget constraint:
	
	\begin{align}
	c_1 \sum_t p_t &= \sum_t p_t e_{1}  \\
	c_1 \left( p_0+ p_1 + p_2 + ... \right) &= \left( e_h p_0+ e_l p_1 + e_h p_2 + ... \right)
	\end{align}
	
	\item And notice

	\begin{align}
	&p_1 = \frac{p_1}{p_0} p_0 = \beta p_0 \\
	&p_2 = \frac{p_2}{p_1} p_1 = \beta \beta p_0 \\
	&... \\
	&p_t =  \beta^t p_0
	\end{align}
	
	\item Let $p_0=1$ (numeraire) and use this in the budget constraint of the consumer to get
	
	\begin{align}
	&c_1 \sum_t \beta^t= e_h (1+\beta^2+\beta^4+...)+ \beta e_l (1+\beta^2+\beta^4+...) \nonumber \\
	&\frac{c_1}{1-\beta}=\frac{e_h}{1-\beta^2}+\frac{\beta e_l}{1-\beta^2} \nonumber \\
	&c_{1}=e_h\frac{1-\beta}{1-\beta^2}+e_l\frac{\beta(1-\beta)}{1-\beta^2}
	\end{align}
	
	\item And likewise
	\begin{align}
	&c_{2}=e_l\frac{1-\beta}{1-\beta^2}+e_h\frac{\beta(1-\beta)}{1-\beta^2}
	\end{align}
	
	\end{itemize}

\subsubsection*{Intuition}

\bit

\item Notice:

\begin{align}
c_1-c_2=\left(e_h-e_l \right) \left( \frac{1-\beta}{1-\beta^2}-\frac{\beta(1-\beta)}{1-\beta^2} \right)=\left(e_h-e_l \right)\frac{(1-\beta)^2}{1-\beta^2}>0
\end{align}

\item Thus, consumption of agent 1 $>$ consumption of agent 2, and this difference is decreasing $\beta$. Reason is that agent 1 gets high endowment before agent 2, which (owing to discounting) makes agent 1 richer in present value terms. 

\eit


\subsection*{Part (d)}

\begin{itemize}
	\item We know from above that

	\begin{align}
	\frac{u^{\prime}(c_{1,t+1})}{u^{\prime}(c_{1,t})}=\frac{u^{\prime}(c_{2,t+1})}{u^{\prime}(c_{2,t})}
	\end{align}

\item Under the given preferences this implies

	\begin{align}
	\frac{c_{1,t}}{c_{1,t+1}}=1 \Rightarrow c_{1,t}=c_{1,t+1}
	\end{align}

\item Now the consolidated budget constraint of each agent reads

\begin{align}
\sum_t \beta^t c_{i,t}= e_l+\beta e_h+ \beta^2 e_l+...\\
\sum_t \beta^t c_{i,t}=\frac{e_l}{1-\beta^2}+\frac{e_h \beta}{1-\beta^2}
\end{align}

\item For agent 1 this implies
\begin{align}
\frac{c_1}{1-\beta}=\frac{e_l}{1-\beta^2}+\frac{e_h \beta}{1-\beta^2} \Rightarrow c_1=e_l\frac{1-\beta}{1-\beta^2}+e_h\frac{ \beta(1-\beta)}{1-\beta^2}
\end{align}

\item Via market clearing this then implies

\begin{align}
c_{2,t=odd}=2e_h-c_1=e_h\frac{2-\beta^2-\beta}{1-\beta^2}-e_l\frac{1-\beta}{1-\beta^2} \\
c_{2,t=even}=2e_l-c_1 =e_l \frac{1-2*\beta^2+\beta}{1-\beta^2}-e_h\frac{ \beta(1-\beta)}{1-\beta^2}
\end{align}

\end{itemize}

\subsubsection*{Intuition}

\bit

\item Notice $c_{1,t=even}=c_{1,t=odd}$ while $c_{2,t=even}<c_{1,t=odd}$, while the present value of consumption of agent 1 equals that of agent 2. 

\item This ows to the fact that agent 1 preferences are such that he/she dislikes consumption variation more than agent 2. Hence, agent 2 will insure agent 1 by taking all the variation in aggregate income.

\eit

\section*{Problem 2}

\subsection*{Part (a)}

\begin{itemize}
	\item A \textbf{sequential equilibrium} is a set of sequences for allocations $\left\{ k_{it}^*, k_{ct}^*, k_{t}^*, n_{it}^*, n_{ct}^*, c_{t}^*, i_{t}^* \right\}_{t=0}^{\infty}$ and prices $\left\{ p_t^*, r^*_t, w^*_t  \right\}_{t=0}^{\infty}$ such that:
	\begin{enumerate}
		\item $\left\{c_{t}^*, i_{t}^*, k_{t}^*  \right\}_{t=0}^{\infty}$ solves the household's problem:
		\begin{align}
		&\max_{\{{c_t, i_t}\}_{t=0}^{\infty}} {\sum_{t=0}^{\infty} \beta^t u(c_t)} \\
		&\text{s.t. } \nonumber \\
		&c_t+p_t^* i_t=r^*_t k_t+w_t^* n_t \nonumber \\
		&k_{t+1}=(1-\delta)k_t+i_t \label{eq:kacc} \nonumber
		\end{align}
		
	\item $\left\{k_{ct}^*, k_{ct}^*  \right\}_{t=0}^{\infty}$ solves problem of consumption good firm:
	
		\begin{align}
		&\max_{k_{ct}, n_{ct}} {A_t^{1-\alpha}k_{ct}^\alpha n_{ct}^{\alpha}-r_t^* k_{ct}-w_{t}^* n_{ct}} \text{ }\forall t
		\end{align}
		
	\item $\left\{k_{it}^*, k_{it}^*  \right\}_{t=0}^{\infty}$ solves problem of the investment good firm:
	
		\begin{align}
		&\max_{k_{it}, n_{it}} {p_t q_t A_t^{1-\alpha}k_{it}^\alpha n_{it}^{\alpha}-r_t^* k_{it}-w_{t}^* n_{it}} \text{ }\forall t
		\end{align}
		
		\item And market clearing (feasibility) holds
		
		\begin{align}
		n^*_{it}+n^*_{ct}=n^*_{t} \label{eq:nclearing} \\
		k^*_{it}+k^*_{ct}=k^*_{t} \label{eq:kclearing} \\
		c^*_t=A_t^{1-\alpha}k^*_{ct}^\alpha n^*_{ct}^{\alpha} \\
		i^*_t=q^*_t A_t^{1-\alpha}k^*_{it}^\alpha n^*_{it}^{\alpha}
		\end{align}
		
	\end{enumerate}
	
\end{itemize}

\subsection*{Part (b)}

\begin{itemize}
	\item Take first order conditions to firms problem to get
	
	\begin{align}
	\alpha A_t ^{1-\alpha} k_{ct}^{\alpha-1} n_{ct}^{1-\alpha}=r_t^* \label{eq:foc1} \\
	(1-\alpha) A_t ^{1-\alpha} k_{ct}^{\alpha} n_{ct}^{-\alpha}=w_t^* \label{eq:foc2} \\
	p_t q_t \alpha A_t ^{1-\alpha} k_{it}^{\alpha-1} n_{it}^{1-\alpha}=r_t^* \label{eq:foc3} \\
	p_t q_t (1-\alpha) A_t ^{1-\alpha} k_{it}^{\alpha} n_{it}^{-\alpha}=w_t^* \label{eq:foc4}
	\end{align}
	
	\item Divide \eqref{eq:foc1} by \eqref{eq:foc2} and \eqref{eq:foc3} by \eqref{eq:foc4} 
	\begin{align}
	\frac{\alpha}{1-\alpha}\frac{n_{ct}}{k_{ct}}=\frac{r^*_t}{w^*_t} \label{eq:nk1} \\
	\frac{\alpha}{1-\alpha}\frac{n_{it}}{k_{it}}=\frac{r^*_t}{w^*_t} \label{eq:nk2}
	\end{align}
	
	\item Equate to get
	
	\begin{align}
	\frac{n_{ct}}{k_{ct}}=\frac{n_{it}}{k_{it}} \label{eq:nk3}
	\end{align}
	
	\item Then divide \eqref{eq:foc3} by \eqref{eq:foc1} and use \eqref{eq:nk3} to get
	\begin{align}
	p_t q_t \left(\frac{k_{it}}{k_{ct}}\right)^{\alpha-1} \left(\frac{n_{it}}{n_{ct}}\right)^{1-\alpha}&=1 \nonumber \\
	\Rightarrow p_t q_t \left( \frac{k_{it}}{n_{it}}\right)^{\alpha-1} \left( \frac{k_{ct}}{n_{ct}}\right)^{1-\alpha}&=1 \nonumber \\
	\Rightarrow p_t &= 1/q_t \label{eq:pq}
	\end{align}

	
\end{itemize}

\subsection*{Part (c)}


\begin{itemize}

\item Start by noting from \eqref{eq:nk3} that $k_{it}/k_{ct}=n_{it}/n_{ct}$ why we can write capital and labor in each sector as the same fraction of total labor and capital:\footnote{To see this note, that $k_{it}/k_{ct}=\frac{s_k k_{t}}{(1-s_k)k_{t}}=\frac{s_k }{(1-s_k)}=n_{it}/n_{ct}=\frac{s_n n_{t}}{(1-s_n)n_{t}}=\frac{s_n }{(1-s_n)}$ why $s_k=s_n$} 
\begin{align}
k_{it}&=sk_{t} \\
n_{it}&=sn_{t} \\
k_{ct}&=(1-s)k_{t} \\
n_{ct}&=(1-s)n_{t} 
\end{align}

\item Use this in \eqref{eq:foc1} to get an expression for $r_t$ 
\begin{align}
r_t^*&=\alpha A_t^{1-\alpha} \left( s k_t \right)^{\alpha-1} \left( s n_t \right)^{1-\alpha} \nonumber \\
\Rightarrow r_t^*&= s^{\alpha-1+1-\alpha} \alpha A_t^{1-\alpha}  k_t ^{\alpha-1} n_t^{1-\alpha} \nonumber \\
\Rightarrow r_t^*&= s^{0} \alpha A_t^{1-\alpha}  k_t ^{\alpha-1} n_t^{1-\alpha} \label{eq:r} 
\end{align}

\item Use this in \eqref{eq:foc2} to get an expression for $w_t$ 
\begin{align}
w_t^*&=(1-\alpha) A_t ^{1-\alpha} \left( s k_{t} \right)^{\alpha} \left( s  n_{t} \right)^{-\alpha} \nonumber \\
\Rightarrow w_t^*&=s^{0} (1-\alpha) A_t ^{1-\alpha} k_{t}^{\alpha}   n_{t} ^{-\alpha} \label{eq:w}
\end{align}

	\item Then plug \eqref{eq:r} and \eqref{eq:w} along \eqref{eq:pq} with  into the budget constraint of the household to get
	\begin{align}
	c_t+\frac{1}{q_t} i_t&=\alpha A_t^{1-\alpha}  k_t ^{\alpha-1} n_t^{1-\alpha} k_t+  (1-\alpha) A_t ^{1-\alpha} k_{t}^{\alpha}   n_{t} ^{-\alpha} n_t \nonumber \\
	\Rightarrow c_t+\frac{1}{q_t} i_t&= A_t^{1-\alpha}  k_t ^{\alpha} n_t^{1-\alpha} \\
	\end{align}
	
\item Hence, the problem of the social planner can be written 
		
		\begin{align}
		&\max_{\{{c_t^*,i_t^* }\}_{t=0}^{\infty}} {\sum_{t=0}^{\infty} \beta^t u(c_t)} \\
		&\text{s.t. } \nonumber \\
		&c_t+\frac{1}{q_t} i_t=A_t^{1-\alpha} k_t^{\alpha} n_t^{1-\alpha} \nonumber \\
		&k_{t+1}=(1-\delta)k_t+i_t \nonumber
		\end{align}
	
	\subsubsection*{Intuition}
	
\item Notice that we have shown that the economy \emph{aggregates}. That is, we started with a disaggreated economy with two sectors with each their production functions. But we have shown that this economy can be represented by one production function (as long as we also correct the price of the investment good by $q_t$ - which can be thought of as the relative productivity of the investment good sector). Also notice that this result was independent of the use of preferences : the result was simply derived from the firms first order conditions.

\item This result is important because it hints at something more general: That we often can represent an economy with many different sectors by \emph{one} aggregate production function.
	
\end{itemize}

\end{document}